\documentclass[a4paper,12pt]{article}
\usepackage[ngerman]{babel}
\usepackage[utf8]{inputenc}
\usepackage{pst-barcode}
\usepackage{booktabs}
\usepackage{makecell}
\usepackage{tikz}
\usetikzlibrary{automata, arrows, fit}
\usepackage{marginnote}
\usepackage{lipsum}
\usepackage{graphicx}
\usepackage{epstopdf}
\usepackage{amsmath, tabu}
\usepackage{amsfonts}
\usepackage{geometry}
\geometry{
	headsep=0.5cm,
	headheight=2.5cm,
	marginparwidth=2cm,
	textheight=22cm,
}
\setlength{\headheight}{120pt}
\usepackage{fancyhdr}
\setlength\parindent{0pt}
\renewcommand{\familydefault}{\sfdefault}
\renewcommand{\arraystretch}{1.2}
\allowdisplaybreaks
\graphicspath{{../figures/}}


\pagestyle{fancy}
\fancyhf{}

\lhead{
	\textit{Humboldt-Universität zu Berlin, Institut der Informatik}\\
	\bigskip
	\textbf{\Large{Algorithmen und Datenstrukturen}}\\
	\bigskip
	Abgabe: 11.05.2020\\
	Blatt 01\\[-1em]
}
\rhead{
	Maximilian Kraska McKone, 573616\\
	Maria Hemmerich\\
	Haoyuan Yan\\[-1em]
}

\newcommand*{\QED}{\hfill\ensuremath{q.e.d.}}
\newcommand{\ex}[1]{\newpage\subsubsection*{Aufgabe #1.}}
\newcommand*{\simpleTable}[2]{
	\begin{tabular}{@{} #1 @{}}
		\toprule
		#2
		\bottomrule
	\end{tabular}
}
\DeclareMathOperator{\sig}{sig}

\newcommand\addvmargin[1]{
  \node[fit=(current bounding box),inner ysep=#1,inner xsep=0]{};
}

\usetikzlibrary{arrows}

\begin{document}
	%%%%%%%%%%%%%%%%%%%% Aufgabe 1 %%%%%%%%%%%%%%%%%%%%%%%%%%%%%%%%%%%%%%%%%%%%%%%%%
	\ex{1}

	\textbf{Funktionen Ordnen}
	\begin{center}
		
	\end{center}

	%%%%%%%%%%%%%%%%%%%% Aufgabe 2 %%%%%%%%%%%%%%%%%%%%%%%%%%%%%%%%%%%%%%%%%%%%%%%%%
	\ex{2}

	\textbf{Eigenschaften der O-Notation}
	\begin{center}
		
	\end{center}

	%%%%%%%%%%%%%%%%%%%% Aufgabe 3 %%%%%%%%%%%%%%%%%%%%%%%%%%%%%%%%%%%%%%%%%%%%%%%%%
	\ex{3}

	\textbf{Pseudocodeanalyse}
	\begin{center}
		1.)
		a.)\newline
		Das Algorithmus 'bar' nimmt als Eingabe ein Array natürlicher Zahlen, wobei sie nicht länger als 2 Elementen sein darf, und geht alle Elemente durch.\newline
		Es wird die größte Differenz zweier Elemente gesucht und ausgegeben.\newline
		
		2.)
		a.)\newline
		Das Algorithmus 'foo' nimmt natürliche Zahlen als Eingabe über 0.\newline
		Falls die Eingabe 1 ist, ist die Ausgabe 2.\newline
		Sonst wird die Funktion angewendet 2*n + foo(n-1) angewendet, und ein Ergebnis der linearen Funktion ausgegeben.
	\end{center}

	%%%%%%%%%%%%%%%%%%%% Aufgabe 4 %%%%%%%%%%%%%%%%%%%%%%%%%%%%%%%%%%%%%%%%%%%%%%%%%
	\ex{4}

	\textbf{Algorithmenentwurf}
	\begin{center}

	4.)
	a.)

	\caption{Algorithmus 'Check identical elements'} \newline
	\text Input: \tab 2 ordered Arrays A and B of length n\text \newline
	\text Output: \tab Boolean value.\text \newline

	\text1:	\tab x:= true\newline \text
	\text2:	\tab 	for i:= 1 to n do\newline \text
	\text3:	\tab 		for j:= 1 to n do\newline \text
	\text4:	\tab 		if A[i] != A[j]\newline \text
	\text5:	\tab 			x:=false\newline \text
	\text6:	\tab 		end if\newline \text
	\text7:	\tab 	end for\newline \text
	\text8:	\tab end for\newline \text
	\text9:	\tab return x\newline \text
		
	\end{center}
\end{document}
